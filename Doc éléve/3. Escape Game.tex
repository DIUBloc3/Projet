\documentclass[a4paper,10pt,addpoints]{exam}
\usepackage{graphicx}
\usepackage{calc}
\usepackage{xcolor}
\usepackage[utf8x]{inputenc}
\usepackage{dirtree}
\usepackage[french]{babel}
\usepackage[T1]{fontenc}
\usepackage{amsfonts,amssymb}
\usepackage[margin=1cm,includeall]{geometry}
\parindent=0cm
\usepackage[french,ruled]{algorithm2e}
\usepackage{listings}
\usepackage{array,multirow,makecell}
\usepackage{colortbl}
\setcellgapes{1pt}
\usepackage{multicol}
\usepackage{wrapfig}
\usepackage{subcaption}
%\usepackage{fancyhdr}
\usepackage{appendix}
\usepackage{enumitem}
\makegapedcells
%\pagestyle{fancy}
%\renewcommand{\headrulewidth}{2pt}
%\renewcommand{\footrulewidth}{1pt}
\pagestyle{headandfoot}
\extraheadheight{1cm}
\headrule
\header{\includegraphics[scale=1.2]{logo.png}}
				{}
				{Première, Spécialité Numérique et Sciences Informatiques}
\extrafootheight{1cm}
\footrule
\footer{\includegraphics[scale=0.20]{cc.png}}
				{Séquence 4, Systèmes d'exploitation}
				{Page \thepage /\numpages}
				
\newenvironment{oneparcheckboxescentering}{
	\begingroup
	\leftskip=80mm plus .5fil%
	\rightskip=0mm plus -.5fil%
	\parfillskip=0mm plus 1fil\relax
	\begin{oneparcheckboxes}
	}{
	\end{oneparcheckboxes}
	\par
	\endgroup
}
\lstset{literate=
	{á}{{\'a}}1 {é}{{\'e}}1 {í}{{\'i}}1 {ó}{{\'o}}1 {ú}{{\'u}}1
	{Á}{{\'A}}1 {É}{{\'E}}1 {Í}{{\'I}}1 {Ó}{{\'O}}1 {Ú}{{\'U}}1
	{à}{{\`a}}1 {è}{{\`e}}1 {ì}{{\`i}}1 {ò}{{\`o}}1 {ù}{{\`u}}1
	{À}{{\`A}}1 {È}{{\'E}}1 {Ì}{{\`I}}1 {Ò}{{\`O}}1 {Ù}{{\`U}}1
	{ä}{{\"a}}1 {ë}{{\"e}}1 {ï}{{\"i}}1 {ö}{{\"o}}1 {ü}{{\"u}}1
	{Ä}{{\"A}}1 {Ë}{{\"E}}1 {Ï}{{\"I}}1 {Ö}{{\"O}}1 {Ü}{{\"U}}1
	{â}{{\^a}}1 {ê}{{\^e}}1 {î}{{\^i}}1 {ô}{{\^o}}1 {û}{{\^u}}1
	{Â}{{\^A}}1 {Ê}{{\^E}}1 {Î}{{\^I}}1 {Ô}{{\^O}}1 {Û}{{\^U}}1
	{Ã}{{\~A}}1 {ã}{{\~a}}1 {Õ}{{\~O}}1 {õ}{{\~o}}1
	{œ}{{\oe}}1 {Œ}{{\OE}}1 {æ}{{\ae}}1 {Æ}{{\AE}}1 {ß}{{\ss}}1
	{ű}{{\H{u}}}1 {Ű}{{\H{U}}}1 {ő}{{\H{o}}}1 {Ő}{{\H{O}}}1
	{ç}{{\c c}}1 {Ç}{{\c C}}1 {ø}{{\o}}1 {å}{{\r a}}1 {Å}{{\r A}}1
	{€}{{\euro}}1 {£}{{\pounds}}1 {«}{{\guillemotleft}}1
	{»}{{\guillemotright}}1 {ñ}{{\~n}}1 {Ñ}{{\~N}}1 {¿}{{?`}}1
	{~} {$\sim$}{1}
}
\title{\textbf{i'm root}\\un escape game pour découvrir la ligne de commande}
\author{Lycée Français de Tananarive}
\date{}

\definecolor{blue}{RGB}{51,131,255}

\begin{document}

\maketitle
\thispagestyle{headandfoot}

\section{Objectif}

\noindent\fcolorbox{black}{gray!30}{\parbox{\linewidth-2\fboxrule-2\fboxsep}
	{Alan Turing est, entre autre, le génial mathématicien qui grâce à une machine déchiffra, lors de la seconde guerre mondiale le code énigma, utilisé par les nazis.\\
	Ce jeu vous propose de retrouver le code caché par un utilisateur dont l'ID est turing au sein d'une Raspberry...	\textbf{Vous avez deux heures !} }}

\section{Qui suis-je ?}
	\begin{questions}
		\boxedpoints
	\question Mon login :
	\makeemptybox{1cm}
	\question Mon mot de passe : 
	\makeemptybox{1cm}

\section{Feuille de route.}
\subsection{première étape : le point de départ}
\setlength{\parindent}{6ex}
\begin{itemize}[label=$\leadsto$, font=\LARGE \color{blue}]
	\item Connectez vous sur la raspberry en utilisant le login/mot de passe communiqué par le professeur. la connection se fait en \textbf{SSH}, soit pas l'intermédiaire d'un terminal, soit en utilisant un logiciel dédié (\textit{Putty}, par exemple).
	\item Chercher un fichier nommé \verb|bienvenue.txt|
	\item Afficher le contenu de ce fichier.
\end{itemize}
	\question [1] Quel est le nom de l'utilisateur qui a caché le mot de passe sur la raspberry ?
	\setlength\answerlinelength{8cm}
	\answerline
	\question [1] Quelle est l'id de l'utilisateur qui a caché le mot de passe sur la raspberry ?
	\answerline
	\question [1] Quelle la première lettre du mot de passe de cet utilisateur ?
	\newline
	\checkboxchar{$\Box$}
	
	\begin{oneparcheckboxescentering}
	\choice a \choice d \choice g \choice 1 \choice c \choice b
	\end{oneparcheckboxescentering}
	\question [1] Quel est le chemin absolu du répertoire dans lequel vous devez vous rendre ?
	\answerline
	\bonusquestion [1] Quelle est la taille du fichier \verb|bienvenue.txt|  ?
	\newline
	
	\begin{oneparcheckboxescentering}
		\choice 720 Mo \choice 3 ko \choice 720 ko \choice 720 o 
	\end{oneparcheckboxescentering}
\subsection{deuxième étape : explorons les alentours}
\begin{itemize}[label=$\leadsto$, font=\LARGE \color{blue}]
	\item Trouvez le fichier \verb|infos.txt|
	\item Affichez intégralement le contenu du fichier
\end{itemize}

		\setlength\answerlinelength{8cm}
		\question [2] Quel est le nom de l'utilisateur dont parle le fichier \verb|infos.tx| ?
		\answerline
		\question [1]A quels groupes appartient cet utilisateur ?
		\answerline
		\question [1] Quels sont les droits des utilisateurs \verb|others| sur le répertoire home de cet utilisateur ?
		\newline
		\checkboxchar{$\Box$}
		
		\begin{oneparcheckboxescentering}
			\choice - - - \choice r - -  \choice r w - \choice r - x \choice r w x 
		\end{oneparcheckboxescentering}
	
		\bonusquestion [1] En quelle année a été écrit l'acticle \textit{A Mathematical Theory of Communication} ?
		\answerline

\subsection{troisième étape : un petit tour chez le voisin}
\begin{itemize}[label=$\leadsto$, font=\LARGE \color{blue}]
	\item Entrez dans le répertoire \verb|/home/$USER/source| où $\$USER$ est le nom de l'utilisateur découvert à l'étape précédente.
	\item Trouvez le fichier \verb|etape3.txt|
	\item Affichez le contenu du fichier en prenant bien soin de pouvoir en lire l'intégralité.
	\item Suivez attentivement les instructions données dans le fichier \verb|etape3.txt|
\end{itemize}
\question [1]Quel est le format du fichier \verb|Code4| ?
\newline

\begin{oneparcheckboxescentering}
	\choice .txt \choice .pdf  \choice .py \choice .odt \choice .sh 
\end{oneparcheckboxescentering}
\question [1] Quelle commande va vous permettre de copier ce fichier dans votre répertoire personnel ?
\answerline
\question[\half] Quelle commande va vous permettre de revenir directement dans votre répertoire personnel ?
\answerline
\question[\half] Quelle commande va vous permettre d'éxécuter le code python contenu dans le fichier \verb|code4| ?
\answerline
\question [1] Quel nom d'utilisateur renvoie l'exécution du script \verb|code4|?
\answerline
\bonusquestion [1] Quelle est la spécialité du mathématicien qui porte ce nom ?
\newline

\begin{oneparcheckboxescentering}
	\choice l'analyse \choice la géomètrie  \choice l'algèbre \choice l'algorithmique 
\end{oneparcheckboxescentering}
\question[1] Quelle commande va vous permettre de vous connecter sous son nom ?
\answerline
\vfill
\subsection{quatrième étape : entrée par effraction}
\begin{itemize}[label=$\leadsto$, font=\LARGE \color{blue}]
	\item Connectez vous sous le nom de l'utilisateur découvert précédemment, normalement vous devez aussi avoir obtenu son mot de passe.
	\item Entrez dans le répertoire \verb|/home/$USER/source| où $\$USER$ est le nom de l'utilisateur découvert à l'étape précédente.
	\item Trouvez le fichier \verb|etape4.txt|
	\item Affichez le contenu du fichier.
\end{itemize}
\question [1] Quels sont les caractères du code de l'utilisateur \verb|turing| que vous possédez déjà ?
\answerline
\question [1] Qu'est ce que \verb|nano| ?
\newline

\begin{oneparcheckboxescentering}
	\choice un éditeur de texte \choice un tableur/grapheur  \choice une calculatrice \choice un logiciel de dessin
\end{oneparcheckboxescentering}
\question [\half] Quelle commande permet de revenir à une session à votre nom ?
\answerline
\question [1] Quelle commande va vous permettre de créer \verb|code5.py|, la copie de \verb|code4.py|?
\answerline
\question [\half] Quelle valeur va prendre la variable \verb|code| dans le script \verb|code5.py| une fois que vous l'aurez modifié ?
\answerline
\subsection{Cinquième étape : modifier pas si simple !  }
\begin{itemize}[label=$\leadsto$, font=\LARGE \color{blue}]
	\item Vous devez être dans votre répertoire et vous devez avoir créer une copie du fichier \verb|code4.py|
	\item Modifiez le fichier \verb|code5.py| que vous venez de créer en suivant les instructions que vous venez de découvrir
	\item Avez vous pensez à vérifier les droits du fichiers \verb|code5.py| avant de tenter de le modifier ?
	\item Vous devriez avoir maintenant le mot de passe de l'utilisateur \verb|turing|.
	\item N'oubliez pas de créer un sous-répertoire \verb|final| dans votre espace personnel, et de lui donner des droits pour un accès à tout le monde, car vous allez devoir y copier des fichiers alors que vous serez sous une session \verb|turing|
\end{itemize}
\question [1] Quel est le type de la variable \verb|maping| dans le script python \verb|code5.py| ?
\newline

\begin{oneparcheckboxescentering}
	 \choice une liste  \choice un dictionnaire \choice un tuple \choice une chaîne de caractères
\end{oneparcheckboxescentering}
\question [1] Quelle commande faut-il entrer pour pouvoir modifier le fichier \verb|code5.py|?
\answerline
\question [1] Quel est l'encodage utilisé dans le fichier \verb|code5.py| ?
\newline

\begin{oneparcheckboxescentering}
	\choice ASCII  \choice unicode \choice latin-1 \choice utf8
\end{oneparcheckboxescentering}
\question [1] De quels types sont les paramètres passés à la fonction \verb|code4(phrase,decalage)| ?
\answerline
\question [1] Quelle combinaison de touches permet de quitter \verb|nano| ?
\answerline
\question [1] Quel est le mot de passe complet de l'utilisateur turing?
\answerline
\question [1] Quelle commande vous a permis de créer le répertoire \verb|final| ?
\answerline
\question [1] Quelle commande vous a permis de changer les droits sur ce dossier ?
\answerline


\subsection{Sixième étape : visite chez le maitre}
\begin{itemize}[label=$\leadsto$, font=\LARGE \color{blue}]
	\item Entrez dans le répertoire personnel de l'utilisateur \verb|turing|. Vous aurez besoin de faire un \verb|su turing| et de rentrer le mot de passe que vous devez déjà posséder.
	\item Les instructions se trouvent dans un ficher \verb|derniere_etape.txt| que vous devrez afficher.
	\item Vous devez trouver deux fichiers, \verb|code_final.txt| et \verb|code_final.sh|
	\item Vous devez maintenant copier dans votre répertoire personnel, dans le dossier \verb|final| que vous avez normalement déjà crée.
	\item N'oubliez pas de rendre le script \verb|code_final.sh| exécutable !
\end{itemize}
\question [1] On considère le fichier \verb|code_final.sh|, dans quel langage est écrit ce script ?
\answerline.
\question [1] Quelle commande va vous permettre de rendre ce script exécutable pour vous ?
\answerline
\question [1] Comment lance-t-on un tel script en ligne de commande ?
\answerline
\question  [2] Quel est le code qui va vous permettre de sortir de la salle que le professeur a fermé à clef ?
\answerline.

\section{Votre résultat}
\begin{center}
	\multirowgradetable{3}[questions]
\end{center}

	\end{questions}
\end{document}
