\documentclass[a4paper,12pt]{article}
\usepackage{graphicx}
\usepackage{calc}
\usepackage{xcolor}
\usepackage[utf8x]{inputenc}
\usepackage{dirtree}
\usepackage[french]{babel}
\usepackage[T1]{fontenc}
\usepackage{amsfonts,amssymb}
\usepackage[margin=1cm,includeall]{geometry}
\parindent=0cm
\usepackage[french,ruled]{algorithm2e}
\usepackage{listings}
\usepackage{array,multirow,makecell}
\usepackage{colortbl}
\setcellgapes{1pt}
\usepackage{multicol}
\usepackage{wrapfig}
\usepackage{subcaption}
\usepackage{fancyhdr}
\usepackage{appendix}
\makegapedcells
\pagestyle{fancy}
\renewcommand{\headrulewidth}{2pt}
\renewcommand{\footrulewidth}{1pt}
\fancyhead[R]{Première, Spécialité Numérique et Sciences Informatiques }
\fancyhead[L]{\includegraphics[scale=0.8]{logo.png}}
\fancyfoot[C]{Séquence 4 : Systèmes d'exploitation}
\fancyfoot[R]{Page \thepage}
\fancyfoot[L]{\includegraphics[scale=0.20]{cc.png}} 
\pagestyle{fancy}
\lstset{literate=
	{á}{{\'a}}1 {é}{{\'e}}1 {í}{{\'i}}1 {ó}{{\'o}}1 {ú}{{\'u}}1
	{Á}{{\'A}}1 {É}{{\'E}}1 {Í}{{\'I}}1 {Ó}{{\'O}}1 {Ú}{{\'U}}1
	{à}{{\`a}}1 {è}{{\`e}}1 {ì}{{\`i}}1 {ò}{{\`o}}1 {ù}{{\`u}}1
	{À}{{\`A}}1 {È}{{\'E}}1 {Ì}{{\`I}}1 {Ò}{{\`O}}1 {Ù}{{\`U}}1
	{ä}{{\"a}}1 {ë}{{\"e}}1 {ï}{{\"i}}1 {ö}{{\"o}}1 {ü}{{\"u}}1
	{Ä}{{\"A}}1 {Ë}{{\"E}}1 {Ï}{{\"I}}1 {Ö}{{\"O}}1 {Ü}{{\"U}}1
	{â}{{\^a}}1 {ê}{{\^e}}1 {î}{{\^i}}1 {ô}{{\^o}}1 {û}{{\^u}}1
	{Â}{{\^A}}1 {Ê}{{\^E}}1 {Î}{{\^I}}1 {Ô}{{\^O}}1 {Û}{{\^U}}1
	{Ã}{{\~A}}1 {ã}{{\~a}}1 {Õ}{{\~O}}1 {õ}{{\~o}}1
	{œ}{{\oe}}1 {Œ}{{\OE}}1 {æ}{{\ae}}1 {Æ}{{\AE}}1 {ß}{{\ss}}1
	{ű}{{\H{u}}}1 {Ű}{{\H{U}}}1 {ő}{{\H{o}}}1 {Ő}{{\H{O}}}1
	{ç}{{\c c}}1 {Ç}{{\c C}}1 {ø}{{\o}}1 {å}{{\r a}}1 {Å}{{\r A}}1
	{€}{{\euro}}1 {£}{{\pounds}}1 {«}{{\guillemotleft}}1
	{»}{{\guillemotright}}1 {ñ}{{\~n}}1 {Ñ}{{\~N}}1 {¿}{{?`}}1
	{~} {$\sim$}{1}
}
\title{Le Bash}
\author{Lycée Français de Tananarive}
\date{}

\begin{document}
	\maketitle
	\thispagestyle{fancy}
\section{Principes Généraux}

\noindent\fcolorbox{black}{gray!30}{\parbox{\linewidth-2\fboxrule-2\fboxsep}
	{Quelque que soit la distribution de Linux, il existe une application appelé $terminal$ qu'on peut lancer et l'interpréteur de commandes avec lequel on interagit est le $bash$. }}
\section{Se connecter sur la Raspberry}
blavle
\section{L'arborescence de fichiers}
\setlength{\parindent}{8ex}Dans un système Unix, on dispose d'une \textbf{arborescence de fichiers} encrée sur $/$ la racine du système de fichiers.
\dirtree{%
	.1 / \DTcomment{répertoire racine}.
	.2 bin \DTcomment{contient les commandes de base}.
	.2 dev \DTcomment{fichier représentant les dispositifs matériels (\textit{devices})}.
	.2 etc \DTcomment{contient les fichiers de configurations sytème}.
	.2 home \DTcomment{contient les répertoires personnels des utilisateurs }.
	.3 alice \DTcomment{répetoire personnel d'Alice}.
	.4 documents.
	.5 \textit{cours.pdf}.
	.5 \textit{notes.odt}.
	.4 Photos.
	.5 \textit{img1.jpg}.
	.5 \textit{img2.jpg}.
	.3 bob.
	.4 documents.
	.5 \textit{encspecs.zip}.
	.2 lib\DTcomment{bibliothéques (\textit{Libraries})}.
	.2 tmp\DTcomment{contient les fichiers temporaires}.
	.2 usr\DTcomment{logiciels installés avec le système}.
	.2 var\DTcomment{Données fréquemment utilisées}.	
}

\section{Le Shell, les commandes de base}
	\subsection{Lister les fichiers}
	la commande $ls$ permet de lister les fichiers (ls = liste)
\begin{lstlisting}[language=bash, frame = single]
# Lister les fichiers contenus dans le répertoire
alice@raspberry: ~$ ls 
cours.pdf note.pdf divers feedback
\end{lstlisting}
$cours.pdf$ et $note.pdf$ sont des fichiers, $divers$ est un répertoire.

Pour obtenir plus d'information, utilisez l'option -l (pour version longue) :
\begin{lstlisting}[language=sh, frame = single]
# Lister les fichiers contenus dans le repertoire
alice@raspberry: ~$ ls  -l
# cours.pdf et note.pdf sont des fichiers
-rw-rw-r--  1 alice nsi   8503 sept. 29 13:58  cours.pdf
-rw-r--r--  1 alice nsi   1503 sept. 20 11:38  note.pdf
# divers est un repertoire, notez le 'd' en debut de ligne
drwx------ 18 alice nsi   4096 nov.  21 19:40  divers
\end{lstlisting}
\subsection{Changer de répertoire}
La commande $cd$ vous permet de changer de répertoire (cd = change directory). Quand vous ouvrez un terminal en mode utilisateur vous êtes dans votre répertoire local (/home/utilisateur).


Dans un système linux la référence au fichier s'appelle un chemin. Dans un chemin le nom des répertoires et des fichiers sont séparés par un "/". Il existe deux types de chemin : absolu et relatif.


Le chemin absolu se base sur la racine de l'arborescence et commence par "/" : ex : /home/utilisateur/<dossier>/<fichier>.
\begin{lstlisting}[language=sh, frame = single]
# vous deplacera dans votre répertoire (/home/utilisateur/dossier)
alice@raspberry: ~$ cd /home/alice/documents  
# Vous entrez dans le répertoire document :
alice@raspberry: ~/documents$   
\end{lstlisting}

Le chemin relatif dépend du répertoire courant où se trouve l'utilisateur Pour se déplacer dans un dossier de l'emplacement courant (par défaut home/utilisateur) vous employez cd suivi du nom du dossier : cd <dossier>. Si vous doutez du nom du dossier, tapez le début de son nom puis appuyez sur la touche Tabulation 

\bigskip
Pour "remonter" d'un répertoire (aller à son parent) on utilise la commande "cd .." 
\begin{lstlisting}[language=sh, frame = single]
alice@raspberry: ~/document$ cd ..  
# Vous remontez vers le répertoire supérieur, votre "home
alice@raspberry: ~$   
\end{lstlisting}
\end{document}
