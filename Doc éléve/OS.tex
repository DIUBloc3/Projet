\documentclass[a4paper,12pt]{article}
\usepackage{graphicx}
\usepackage{calc}
\usepackage{xcolor}
\usepackage[utf8x]{inputenc}
\usepackage{dirtree}
\usepackage[french]{babel}
\usepackage[T1]{fontenc}
\usepackage{amsfonts,amssymb}
\usepackage[margin=1cm,includeall]{geometry}
\parindent=0cm
\usepackage[french,ruled]{algorithm2e}
\usepackage{listings}
\usepackage{array,multirow,makecell}
\usepackage{colortbl}
\setcellgapes{1pt}
\usepackage{multicol}
\usepackage{wrapfig}
\usepackage{subcaption}
\usepackage{fancyhdr}
\usepackage{appendix}
\makegapedcells
\pagestyle{fancy}
\renewcommand{\headrulewidth}{2pt}
\renewcommand{\footrulewidth}{1pt}
\fancyhead[R]{Première, Spécialité Numérique et Sciences Informatiques }
\fancyhead[L]{\includegraphics[scale=0.8]{logo.png}}
\fancyfoot[C]{Séquence 4 : Systèmes d'exploitation}
\fancyfoot[R]{Page \thepage}
\fancyfoot[L]{\includegraphics[scale=0.20]{cc.png}} 
\pagestyle{fancy}

\title{Système d'exploitation}
\author{J. ~\textsc{Devalette}}
\date{}

\begin{document}
	\maketitle
	\thispagestyle{fancy}
\section{Principes Généraux}
\noindent\fcolorbox{black}{gray!30}{\parbox{\linewidth-2\fboxrule-2\fboxsep}
	{Un système d'exploitation est un programme ou un ensemble de programmes dont le but est de gérer les ressources matérielles et logicielles d'un ordinateur }}

\section{L'arborescence de fichiers}
\dirtree{%
	.1 / \DTcomment{répertoire racine}.
	.2 bin \DTcomment{contient les commandes de base}.
	.2 etc \DTcomment{contient les fichiers de configurations sytème}.
	.2 home \DTcomment{contient les répertoires personnels des utilisateurs }.
	.3 Alice \DTcomment{répetoire personnel d'Alice}.
	.4 Documents.
	.5 \textit{cours.pdf}.
	.5 \textit{notes.odt}.
	.4 Photos.
	.5 \textit{img1.jpg}.
	.5 \textit{img2.jpg}.
	.3 Bob.
	.4 Documents.
	.5 \textit{encspecs.zip}.
	.2 tmp\DTcomment{contient les fichiers temporaires}.
}

\section{Le Shell}
\begin{lstlisting}[language=bash]
alice$ id
\end{lstlisting}
\end{document}
